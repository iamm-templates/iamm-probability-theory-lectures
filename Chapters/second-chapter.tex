\part{Производящее пространство элементарных событий. Случайные величины и векторы}
\section{Аксиомы теории вероятностей. Вероятностное пространство}
\begin{definition}
	Пусть $\Omega = \{\omega\}$. Набор подмножества $\Omega$ $\mathcal{A}$ называется алгеброй, если
	\begin{enumerate}
		\item $\Omega \in \mathcal{A}$
		\item $\left.
			      \begin{aligned}
				      A \subset \Omega, B \subset \Omega \\
				      A \in \mathcal{A}, B \in \mathcal{A}
			      \end{aligned}
			      \right\} \Rightarrow A + B \in \mathcal{A}, A \cdot B \in \mathcal{A}$
		\item $\left.
			      \begin{aligned}
				      A \subset \Omega \\
				      A \in \mathcal{A}
			      \end{aligned}
			      \right\} \Rightarrow \overline{A} \in \mathcal{A}$
	\end{enumerate}
\end{definition}
\begin{definition}
	Набор подмножеств исходного множества $\Omega = \{\omega\}$ --- $\mathcal{F}$ называется $\sigma$-алгеброй, если он является алгеброй и:
	\[
		A_1, A_2, \dots, A_i, \dots A_k \subset \Omega \forall k,
	\]
	\[
		A_i \in \mathcal{F} \Rightarrow \sum\limits_{i=1}^\infty A \in \mathcal{F}, \prod\limits_{i=1}^\infty A_i \in \mathcal{F}.
	\]
\end{definition}
\subsection{Аксиомы Колмогорова}
$\big<\Omega, \mathcal{F}, \Prob \big>$ --- измеримое пространство.
\begin{enumerate}
	\item \textbf{Аксиома алгебры событий.} Заданы множества элементарных событий $\Omega = \{\omega\}$. На $\Omega$ выделена $\sigma$-алгебра $\mathcal{F}$, её элементы --- случайные события.
	\item \textbf{Аксиома существования алгебры событий.} Любому случайному событию $A \in \mathcal{F}$ сопоставлено неотрицательное число, называемое вероятностью этого события, $\forall A \in \mathcal{F}: \Prob(A) \geq 0$.
	\item \textbf{Аксиома нормированности.} $\Prob(\Omega) = 1$.
	\item \textbf{Аксиома аддитивности вероятности.} Если $A, B \in \mathcal{F}, A \times B = \emptyset$ $\Rightarrow$ $\Prob(A + B) = \Prob(A) + \Prob(B)$. \\
	      \begin{conclusion}
		      $A_1, \dots, A_n \in \mathcal{F}$: $A_i \times A_j \not = \emptyset$, $i \not= j$, $i, j = \To n$ $\Rightarrow$ $\Prob(\sum\limits_{i=1}^n A_i) = \sum\limits_{i=1}^n \Prob(A_i)$.
	      \end{conclusion}
	\item $A_1, \dots, A_n, \dots, \underset{i \not= j, i,j = 1, 2, \ldots}{A_i \in \mathcal{F} : A_{ij}} = \emptyset$ (попарно несовместны) $\Rightarrow$ $\Prob(\sum\limits_{i=1}^\infty A_i) = \sum\limits_{i=1}^\infty \Prob(A_i)$. \\ $\Prob$ -- нормированная счётно-аддитивная мера
\end{enumerate}
Рассмотрим монотонную случайную последовательность событий $A_1, \dots, A_n, \dots$
\setcounter{equation}{0}
\begin{equation}\label{2-1-1}
	A_1 \subset A_2 \subset \dots \subset A_n \subset \dots, \overset{\forall n \in \mathbb{N}}{A_n \subset A_{n+1}}
\end{equation}
\begin{equation}\label{2-1-2}
	A_1 \supset A_2 \supset \dots \supset A_n \supset \dots, \overset{\forall n \in \mathbb{N}}{A_n \supset A_{n+1}}
\end{equation}
Тогда
\[ A = \sum\limits_{i=1}^\infty A_i = \lim\limits_{n \to \infty} A_n \text{--- предел (\ref{2-1-1})}.
\]
\[ A = \prod\limits_{i=1}^\infty A_i = \lim\limits_{n \to \infty} A_n \text{--- предел (\ref{2-1-2})}.
\]
\begin{definition}
	Функция событий $Q(A)$ называется непревывной, если для любой монотонной последовательности случайных событий выполняется равенство
	\[
		\lim\limits_{n \to \infty} Q(A_n) = Q(\lim\limits_{n \to \infty} A_n)
	\]
\end{definition}
\begin{enumerate}
	\setcounter{enumi}{4}
	\item[$5'.$] \textbf{Аксиома непрерывности.} Пусть $A_n$ --- монотонно убывающая последовательность случайных событий. \\ $[A_n \downarrow \emptyset] \Leftrightarrow A_{n+1} \subset A_n, n = 1, 2, \dots$ ($\prod\limits_{i=1}^n A_n = \emptyset$ --- предел невозможен). Тогда
	      \[
		      \lim\limits_{n \to \infty} \Prob(A_n) = 0.
	      \]
\end{enumerate}

\begin{theorem}
	Вероятность является непрерывной функцией событий.
\end{theorem}
\begin{proof}
	\begin{description}[leftmargin = 0cm]
		\item Из аксиомы (5) $\Rightarrow$ (5') \\
		      Пусть $B_n \downarrow \emptyset$, $A_n = B_n \cdot \overline{B_{n+1}}$, где $n = 1,2, \dots$ \\
		      \[
			      B_1 = \sum\limits_{k=1}^n A_k, B_n = \sum\limits_{k=n}^\infty A_k,
		      \]
		      \[
			      \Prob(B_1) = \Prob(\sum\limits_{k=1}^\infty A_k) = \sum\limits_{k=1}^\infty \Prob(A_k)
		      \]

		      \[ \Prob(B_n) = \Prob(\sum\limits_{k=n}^\infty A_k) = \underbrace{\sum\limits_{k=n}^\infty \Prob(A_k)}_{n \to \infty} \to 0 \]

		      \[ \lim\limits_{n \to \infty} \Prob (B_n) = 0\]
		\item Из аксиомы (5') $\Rightarrow$ (5) \\
		      Пусть имеется множество попарно несовместных событий ${\{A_n\}}_{n = 1, 2, \dots}$.
		      \[ B_n = \sum\limits_{k=n+1}^\infty A_k, A = \sum\limits_{k=1}^\infty A_k \]
		      \[ A = A_1 + \ldots + A_n + B_n \]
		      \[ \Prob(A) = \sum\limits_{k=1}^n \Prob(A_k) + \underbrace{\Prob(B_n)}_{\to 0} \]
		      \[ B_{n+1} \subset B_n \]
		      \[ B_n \Rightarrow A_i, i > n \Rightarrow A_{i+1} \text{ не наступило} \Rightarrow \text{не наступило } B_i, \dots \]
		      \[ \prod\limits_{i=1}^\infty B_i = 0 \]
	\end{description}
\end{proof}
\section{Свойства вероятности}
\begin{enumerate}
	\item Вероятность невозможного события $\Prob(\emptyset) = 0$
	      \begin{proof}
		      $ \emptyset+ \Omega = \Omega $
		      \[
			      1 = \Prob(\Omega) = \Prob(\emptyset + \Omega) \overset{(\text{акс. } 4)}{=} \Prob(\emptyset) + \underbrace{\Prob(\Omega)}_{= 1} = 1
		      \]
	      \end{proof}
	\item $\Prob(\overline{A}) = 1 - \Prob(A)$
	      \begin{proof}
		      \[
			      1 = \Prob(\Omega) = \Prob(A + \overline{A}) \overset{(\text{акс. } 4)}{=} \Prob(A) + \Prob(\overline{A}) = 1
		      \]
	      \end{proof}
	\item $A \subset B$ $\Rightarrow$ $\Prob(A) \leqslant \Prob(B)$
	      \begin{proof} $B = A \times B + \overline{A} \times B \overset{A \subset B}{=} A + \overline{A} \times B$
		      \[
			      \Prob(B) = \Prob(A + \overline{A} \times B) \overset{(\text{акс. } 4)}{=} \Prob(A) + \Prob(\overline{A} \times B)
		      \]
	      \end{proof}
	\item $\Prob(A + B) = \Prob(A) + \Prob(B) - \Prob(A \times B)$
	      \begin{proof}
		      \[
			      A + B = A + \overline{A} \times B, B = A \times B + \overline{A} \times B \Rightarrow
		      \]
		      \[
			      \Prob(A + B) \overset{(\text{акс. } 4)}{=} \Prob(A) + \Prob(\overline{A} \times B)
		      \]
		      \[
			      \Prob(B) = \Prob(A \times B) + \Prob(\overline{A} \times B)
		      \]
	      \end{proof}
	\item $\Prob(A+B) \leqslant \Prob(A) + \Prob(B)$
	      \begin{proof}
		      \[
			      \Prob(\sum\limits_{k=1}^n A_k) \leqslant \sum\limits_{k=1}^n \Prob(A_k)
		      \]
		      \[
			      \Prob(\sum\limits_{k=1}^\infty A_k) \leqslant \sum\limits_{k=1}^\infty \Prob(A_k)
		      \]
	      \end{proof}
	\item $A_1, \dots, A_n$:
	      \[
		      \begin{split}
			      \Prob(\sum\limits_{k=1}^\infty A_k) = \sum\limits_{k=1}^\infty \Prob(A_k) - \sum\limits_{k=1}^{n-1} \sum\limits_{i=k+1}^n \Prob(A_k \times A_j) + \ldots \\
			      \ldots - (-1)^{n-1} \cdot \Prob(\prod\limits_{k=1}^n A_k)
		      \end{split}
	      \]
	      Пусть $B = \sum\limits_{k=1}^{n+1} A_k$. Тогда
	      \[
		      \Prob(\sum\limits_{k=1}^{n+1} A_k) = \Prob(A_{k+1} + \sum\limits_{k=1}^n A_k)
	      \]
	\item \textbf{Теорема.} Пусть имеется $k$ попарно несовместных и составляющих полную группу благоприятных к событию $A$ исходов из всех исходов $n$, $\{ E_1, \dots, E_n\}_{\underset{i, j = 0, 1, \ldots}{E_i \times E_j = \emptyset, i \not = j}}$, $\sum\limits_{n=1}^n E_i = \Omega$. Тогда
	      \[
		      \Prob(A) = \frac{k}{n}
	      \]
	      \begin{proof}
		      $A = \sum\limits_{s=1}^k E_{i_s}$
		      \[
			      1 = \Prob(\sum\limits_{i=1}^n E_i) \overset{(\text{акс. } 4)}{=} \sum\limits_{i=1}^n \Prob(E_i) = n \cdot \Prob(E_i), \Prob(E_i) = \frac{1}{n}
		      \]
		      \[
			      \Prob(A) = \Prob(\sum\limits_{s=1}^k E_{i_s}) \overset{(\text{акс. } 4)}{=} \sum\limits_{s=1}^k \Prob(E_{i_s}) = \frac{k}{n}
		      \]
		      \[
			      \Omega = \{E_1, \dots, E_n\}, \mathcal{F} = \{A = E_{i_1} + \ldots + E_{i_k}\}, k \leqslant n
		      \]
		      \[
			      \Prob(A) = \frac{k}{n}
		      \]
	      \end{proof}
	      \setcounter{enumi}{7}
	\item $0 \leqslant \Prob(A) \leqslant 1, A \in \mathcal{F}$
	      \[
		      0 \leqslant \Prob(\overline{A}) = 1 - \Prob(A) \Rightarrow \Prob(A) \leqslant 1
	      \]
	      Для условных вероятностей:
	      \[
		      \Prob(A|B) \overset{def}{=} \frac{\Prob(A \times B)}{\Prob(B)}
	      \]
	      $\Prob(B) > 0$
	      \[
		      \Prob(\Omega | B) = \frac{\Prob(\Omega \times B)}{\Prob(B)} = \frac{\Prob(B)}{\Prob(B)} = 1.
	      \]
	      Аксиома нормированности выполнена. \\
	      Пусть $A_1, \ldots, A_n, \ldots$ --- не более чем счётный набор, $A_i \times A_j = \emptyset, i \not = j; i,j = \To n$ \\
	      Пусть также имеется событие $B$, $\Prob(B) > 0$.
	      \[
		      \Prob(\sum\limits_{n} A_n | B) = \frac{\Prob((\sum\limits_{n} A_n) \times B)}{\Prob(B)} = \frac{\sum\limits_{n} \Prob(A_n \times B_n)}{\Prob(B)} =
	      \]
	      \[
		      \overset{(\text{акс. } 4, 5)}{=} \sum\limits_{n} \Prob(A_n | B)
	      \]

	\item $\mathcal{F}_* = \{\emptyset, \Omega\}$ --- наименьшая $\sigma$-алгебра, $\mathcal{F}^* = \{A: A \subseteq \Omega \}$ --- наибольшая $\sigma$-алгебра. \\
	      Пусть имеется $n$ испытаний в эксперименте с подбрасыванием монет. Построим измеримое пространство

	      \[
		      \Omega = \{ \text{О, Р} \}, \mathcal{F}^{*} : \{ \text{О} \}, \{ \text{Р} \}, \{ \text{О + Р} \}, \emptyset
	      \]

	      Зададим вероятности событий (О --- орёл, Р --- решка)
	      \[
		      \Prob(\text{О}) = p, \Prob(\text{Р}) = q: p + q = 1.
	      \]
	      Создано вероятностное пространство. \\
	      В случае несчётного пространства элементарных событий набор всех подмножеств элементарных событий и построенная $\sigma$-алгебра не будет удовлетворять испытанию, чтобы задать вероятность на этом множестве. \\
	      Можем основываться на определённом наборе подмножеств $\Omega$.
	      \begin{definition}
		      Наименьшая $\sigma$-алгебра, содержащая заданный класс подмножеств $\Omega$, называется $\sigma$-алгеброй, порождаемой этим классом.
	      \end{definition}

	      \begin{example}
		      $\Omega = [0, 1]$. Пусть $\sigma$-алгебра $\mathcal{F}$ составляет все подмножества, для которых можно определить длину, $\{\omega\}$ --- точки отрезка $[0, 1]$.
		      \[
			      A \in \mathcal{F} : \Prob(A) = \mu (A) \text{ --- длина некоторого отрезка, мера}
		      \]
		      При не более чем счётном пространстве событий мы говорим о дискретных пространствах. В противном случае --- о непрерывно измеримых.
	      \end{example}
\end{enumerate}
\section{Способы задания вероятностных мер на измеримом пространстве $(\mathbb{R}^1, \mathcal{B} (\mathbb{R}^1) )$}
$\mathbb{R}^1 = (-\infty; +\infty)$ --- вещественная прямая. Рассмотрим на $\mathbb{R}^1$ интервал такого вида:
\[
	\{ (a, b] = \{x \in \mathbb{R}^1 | a < x \leqslant b \} \}
\]
\[
	- \infty \leqslant a < b < +\infty
\]
Рассмотрим множество таких интервалов.
\begin{definition}
	Наименьшая $\sigma$-алгебра, содержащая систему вида (*), называется борелевской алгеброй множеств вещественной прямой.
\end{definition}
\begin{definition}
	Борелевские множества -- элементы $\mathcal{B}(\mathbb{R}^1)$. Они же и являются событиями.
\end{definition}
Заметим, что интервалы
\[
	(a, b) = \sum\limits_n (a, b - \frac{1}{n}]
\]
Отрезки:
\[
	[a, b] = \prod\limits_n (a - \frac{1}{n}, b]
\]
Например, точка $a$:
\[
	\{ a \} = \prod\limits_n (a - \frac{1}{n}, a]
\]
Оставшиеся на рассмотрении $(a, b]$ (и др. виды) ведут к той же борелевской алгебре. \\

Построим вероятностное пространство на основе этого измеримого пространства. Пусть $\Prob$ -- вероятность, заданная на борелевском множестве. Возьмём борелевское множество:
\[
	B = (- \infty, x], x \in \mathbb{R}^1
\]
Положим
\begin{equation}\label{2-3-1}
	F (x) = \Prob \{ (- \infty, x] \}
\end{equation}
Так определённая функция, оказывается, обладает следующими свойствами:
\begin{enumerate}
	\item $F(x)$ монотонно не убывает на $\mathbb{R}^1$.
	      \begin{proof}
		      Пусть $x_1 < x_2$. Тогда соответствующее событие
		      \[
			      (-\infty, x_1] \subset (-\infty, x_2] \Rightarrow \Prob \{ (-\infty, x_1] \} \leqslant \Prob \{ (-\infty, x_2] \}
		      \]
		      Итак, функция монотонно неубывающая
		      \[
			      F(x_1) \leqslant F(x_2), \forall x_1, x_2 \in \mathbb{R}^1, x_1 < x_2
		      \]
	      \end{proof}
	\item Пусть $\lim\limits_{x \to -\infty} F(x) = F(-\infty)$, $\lim\limits_{x \to +\infty} F(x) = F(+\infty)$, тогда $F(-\infty) = 0, F(\infty) = 1$.
	      \begin{proof}
		      В силу монотонности предел функции существует. Докажем, что $\underset{n = 1, 2, \ldots}{F(-n)} \underset{n \to \infty}{\rightarrow} 0.$ \\
		      Рассмотрим последовательность событий:
		      \[
			      \{ B_n = (-\infty, -n] \}
		      \]
		      \[
			      \left.
			      \begin{split}
				      B_{n+1} \subset B_n \\
				      \lim\limits_{n \to \infty} = \prod\limits_n B_n = \emptyset
			      \end{split}
			      \right\}
			      B_n \downarrow \emptyset
		      \]
		      Для этой последовательности (по акс. непрерывности)
		      \[
			      \lim\limits_{n \to \infty} \underbrace{\Prob(B_n)}_{F(-n)} = 0
		      \]
		      %TODO: для второго предела
	      \end{proof}
	\item $F(x)$ непрерывна справа и имеет предел слева $\forall x \in \mathbb{R}^1$
	      \[
		      \lim\limits_{y \uparrow x} F(y) = F(x - 0), \lim\limits_{y \downarrow x} F(y) = F(x + 0)
	      \]
	      Тогда $\forall x \in \mathbb{R}^1: F(x+0) = F(x)$, $F(x) - F(x-0) \geqslant 0$.
	      \begin{proof}
		      Рассмотрим монотонную последовательность $x_n \downarrow x$. Введём последовательность событий
		      \[
			      \{ \underset{n = 1, 2, \ldots}{B_n} = (-\infty, x_n] \}, B = (-\infty, x]
		      \]
		      Рассмотрим предел $\{ B_n \}_{n = 1, 2, \ldots}$
		      \[
			      \prod\limits_n B_n = B
		      \]
		      Перейдём к вероятностям:
		      \[
			      \lim\limits_{n \to \infty} \Prob(B_n) = \Prob(\lim\limits_{n \to \infty} B_n) = \Prob(B)
		      \]
		      \[
			      \Prob(B) = F(x) \Rightarrow \lim\limits_{n \to \infty} F(x_n) = F(x)
		      \]
	      \end{proof}
\end{enumerate}
\begin{definition}
	Любая функция, удовлетворяющая условиям (1, 2, 3), называется функцией распределения на вещественной прямой.
\end{definition}
\begin{figure}[ht] %TODO
	\centering
	\def\svgwidth{16em}
	\input{Images/image.pdf_tex}
	\caption{Какой-то рисунок, который ещё не нарисовали.}
\end{figure}
Оказывается, что распределение вероятности имеет не более чем счётное количество скачков (разрывов функции).
Функция распределения может иметь не более
\begin{itemize}
	\item 1 скачка размером $\frac{1}{2}$,
	\item 2 скачков размером $\frac{1}{4}$,
	\item n скачков размером $\frac{1}{2^n}$.
\end{itemize}
Функция распределения непрерывна $\Leftrightarrow$ вероятность точечного события равна нулю.
\[
	\forall x \in \mathbb{R}^1: F(x) \text{ непрерывна } \Leftrightarrow \Prob(\{ x \})
\]
\begin{proof}
	\[
		\begin{split}
			F(x) - F(x-0) = \lim\limits_{n \to \infty} (F(x) - F(x - \frac{1}{n})) = \\
			= \lim\limits_{n \to \infty} [\{ \Prob(-\infty, x] - \Prob(-\infty, x - \frac{1}{n}] \}]
		\end{split}
	\]
	Представим как сумму двух несовместных событий
	\[
		(-\infty, x - \frac{1}{n}] + (x - \frac{1}{n}, x]
	\]
	\[
		\Prob \{ (-\infty, x] \} = \Prob \{ (-\infty, x - \frac{1}{n} ] \} + \Prob \{ (x - \frac{1}{n}, x ]
	\]
	\[
		\lim\limits_{n \to \infty} \Prob \{ (x - \frac{1}{n}, x] \} = \Prob \{ \prod\limits_n (x - \frac{1}{n}, x] \} = \Prob \{ x \}.
	\]
	Итак, $F(x) - F(x-0) = \Prob \{ x \}$.
\end{proof}
\[
	\{ (a, b] \}, -\infty \leqslant a < b < +\infty
\]
В зависимости от выбора вида интервала свойства (1, 2, 3) изменяются подобно изменению интервала.
\begin{theorem}
	Пусть $F(x)$ --- некоторая функция распределения. Тогда на измеримом пространстве $(\mathbb{R}^1, \mathcal{B}(\mathbb{R}^1))$ $\exists \ !$ вероятностная мера $P$, такая, что
	\[
		P: -\infty \leqslant a < b < +\infty
	\]
	\begin{equation}\label{2-3-2}
		P \{ (a, b] \} = F(b) - F(a)
	\end{equation}
	Соотношения (\ref{2-3-1}) и (\ref{2-3-2}) устанавливают взаимно однозначное соответствие между функцией распределения и вероятностной мерой.
\end{theorem}
\begin{theorem}
	(Каратеодори). Пусть имеется вероятностное пространство в широком смысле $(\Omega, \mathcal{A}, P)$. Тогда существует единственная вероятностная мера $Q$ на $\sigma$-алгебре $\mathcal{F} = \sigma(\mathcal{A})$ --- порождённая $\mathcal{A}$, что:
	\[
		Q(A) = P(A), A \in \mathcal{A}.
	\]
\end{theorem}
\begin{conclusion}
	Любое вероятностное пространство в широком смысле автоматически определяет вероятностное пространство.
\end{conclusion}
Из этого следует, что для определения вероятности достаточно задать вероятности интервалов. \\
Рассмотрим алгебру, элементами которой является сумма непересекающихся интервалов.
\[
	\mathcal{A} : A = \sum\limits_{k = 1}^n (a_k, b_k], a_k < b_k
\]
На этих множествах определим функцию множеств
\[
	\Prob_0 (A) = \sum\limits_{k = 1}^n [ F(b_k) - F(a_k) ] \text{ (вероятности интервалов)}
\]
Проверим выполнение аксиом
\begin{itemize}
	\item Акс. 2, 3, 4 выполняются, что почти очевидно.
	\item Акс. 5 также выполняется, см [1]. %TODO: bibtex
\end{itemize}
\[
	\sigma(\mathcal{A}) = \mathcal{B}(\mathbb{R}^1)
\]
Итак, для любой функции распределения существует единственная вероятностная мера. Таким образом строим вероятностное пространство $(\mathbb{R}^1, \mathcal{B} (\mathbb{R}^1), P)$.

\subsection{Меры на измеримых пространствах}
\subsubsection{Дискретные вероятностные меры}
Пусть $F(x)$ --- функция распределения $x_1 < x_2 < \ldots < \ldots$
\[
	\Delta F(x) = F(x) - F(x-0),
\]
\[
	\Prob(x_k) = \Delta F(x_k), k = 1, 2, \ldots
\]
\[
	\sum\limits_k \Prob(x_k) = 1
\]
Представим иллюстрацию ситуации
\begin{figure}[ht] %TODO
	\centering
	\def\svgwidth{16em}
	\input{Images/image.pdf_tex}
	\caption{Какой-то рисунок, который ещё не нарисовали.}
\end{figure}

Такой набор чисел называется дискретным законом распределения вероятностей на вещественной прямой.
\begin{example}
	$\{ \Prob_k = \frac{1}{N} \}$ --- дискретный равномерный закон, $k = 1, \ldots, N$
\end{example}
\begin{example}
	$\{ p_1, p_2 \} : p_1 = 1 - p_2$ --- бернуллиевское дискретное распределение (часто обозначают как $p, q$).
\end{example}
\begin{example}
	${ \{ {\Prob}_{m}^{ (n) } \} }_{m = 0, 1, \ldots, n}$, $n$ --- число испытаний, $p$ --- вероятность появления успеха в каждом испытании. \\
	$\Prob_m^{(n)} = C_n^m \cdot p^m \cdot (1 - p)^{n-m}$ --- биномиальное дискретное распределение.
\end{example}
\begin{example}
	$\Prob_m = \frac{a^m}{m!} \cdot e^{-a}, a \in \mathbb{R}, a > 0, m = 0, 1, \ldots$ --- дискретный закон распределения Пуассона.
\end{example}
\subsubsection{Абсолютно непрерывные вероятностные меры}
Пусть $F(x)$ непрерывна $\forall x \in \mathbb{R}$, при этом существует вещественная неотрицательная кусочно-непрерывная функция плотности распределения вероятностей $f(x)$:
\[
	f(x) \geqslant 0 : F(x) = \int\limits_{-\infty}^x f(t) dt
\]
В этом случае
\[
	\Prob \{ (a, b] \} = \int\limits_a^b f(x) dx
\]
Очевидно, что если $x$ --- точка непрерывности $f(x)$, $x \in \mathbb{R}$, то
\[
	F'(x) = f(x)
\]
Представим данную ситуацию
\begin{figure}[ht] %TODO
	\centering
	\def\svgwidth{16em}
	\input{Images/image.pdf_tex}
	\caption{Какой-то рисунок, который ещё не нарисовали.}
\end{figure}
\begin{figure}[ht] %TODO
	\centering
	\def\svgwidth{16em}
	\input{Images/image.pdf_tex}
	\caption{Какой-то рисунок, который ещё не нарисовали.}
\end{figure}

Плотностью распределения может быть любая кусочно-непрерывная, вещественно значимая функция $f(x)$, удовлетворяющая условию нормировки
\[
	\int\limits_{-\infty}^{+\infty} f(x) dx = 1
\]
\begin{example}
	Равномерное распределение на отрезке $[a, b]$, $a < b$
	\[
		f(x) = \frac{1}{b-a}, a \leqslant x \leqslant b
	\]
	Если $x < a$ или $x > b$, тогда
	\[
		f(x) = 0.
	\]
	Итак,
	\[
		f(x) =
		\begin{cases}
			\frac{1}{b-a}, a \leqslant x \leqslant b \\
			0, x < a \text{ или } x > b
		\end{cases}
	\]
\end{example}
\begin{example} Распределение Гаусса (нормальное)
	\[
		f(x) ~ N(a, \sigma), a \in \mathbb{R}, \sigma > 0
	\]
	Тогда
	\[
		f(x) = \underbrace{\frac{1}{\sqrt{2 \pi } \sigma}}_{\underset{\text{множитель}}{\text{норм.}}}\cdot e ^{-\frac{(x-a)^2}{2 \sigma^2}}
	\]
	Изобразим на графике
	\begin{figure}[ht] %TODO
		\centering
		\def\svgwidth{16em}
		\input{Images/image.pdf_tex}
		\caption{Какой-то рисунок, который ещё не нарисовали.}
	\end{figure}
\end{example}
\begin{example} $\Gamma$-рапределение \\
	Здесь в нормированном множителе для плотности участвует $\Gamma$-функция
	\[
		f(x) =
		\begin{cases}
			0, x \leqslant 0 \\
			\frac{\alpha^\lambda}{\Gamma(\lambda)} \cdot x^{\lambda - 1} \cdot e^{-\alpha x}, x > 0
		\end{cases}
	\]
	$\alpha$ --- параметр масштаба, $\lambda$ --- параметр формы. \\
	Экспоненциальное (показательное) распределение получаем при $\lambda = 1$:
	\[
		f(x) =
		\begin{cases}
			\alpha e^{-\lambda x}, x \geqslant 0 \\
			0, x < 0
		\end{cases}
	\]
	Изобразим функцию плотности вероятности $f(x)$ экспоненциального распределения
	\begin{figure}[ht] %TODO
		\centering
		\def\svgwidth{16em}
		\input{Images/image.pdf_tex}
		\caption{Какой-то рисунок, который ещё не нарисовали.}
	\end{figure}
\end{example}
\subsubsection{Сингулярные вероятностные меры на $\mathbb{R}^1$}
Оказывается, $F(x)$ может быть непрерывной, но не иметь плотности. \\
$F(x)$ --- непрерывная функция распределения, функции плотности вероятности $f(x)$ не существует.
\begin{example} Канторова функция.
	\[
		F(x) =
		\begin{cases}
			0, x \leqslant 0 \\
			1, x \geqslant 1
		\end{cases}
	\]
	Представим, как ведёт себя функция распределения при $x \in (0, 1)$
	\begin{figure}[ht] %TODO
		\centering
		\def\svgwidth{16em}
		\input{Images/image.pdf_tex}
		\caption{$F_1(x)$ --- первое приближение Канторовой функции.}
	\end{figure}
	\begin{figure}[ht] %TODO
		\centering
		\def\svgwidth{16em}
		\input{Images/image.pdf_tex}
		\caption{$F_2(x)$ --- второе приближение Канторовой функции.}
	\end{figure}

	$F_n(x) \underset{n \to \infty}{\rightarrow} F(x)$ --- Канторова функция. \\
	Длина интервалов, на которой Канторова функция должна быть постоянна:
	\[
		\frac{1}{3} + \frac{2}{9} + \frac{4}{27} + \ldots = \frac{1}{3} \cdot \sum\limits_{n = 0}^\infty (\frac{2}{3})^n = 1.
	\]
	$f(x)$ почти всюду обращается в 0, за исключением, может быть, множеств меры нуль. \\
	Функция называется сингулярной, поскольку она сингулярна по отношению к мере Лебега.
\end{example}
\begin{theorem} (Лебега.)
	Любая функция распределение в таком измеримом пространстве может быть представлена в виде:
	\[
		F(x) = p_1 F_1(x) + p_2 F_2(x) + p_3 F_3(x)
	\]
	где $p_i \geqslant 0, \sum\limits_{i=1}^3 p_i = 1$,
	\begin{itemize}
		\item $F_1 (x)$ --- дискретная функция распределения
		\item $F_2 (x)$ --- абсолютно непрерывная функция распределения
		\item $F_3 (x)$ --- сингулярная функция распределения
	\end{itemize}
\end{theorem}
\section{Случайные величины}
