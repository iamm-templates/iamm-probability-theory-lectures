\part{Производящее пространство элементарных событий. Случайные величины и векторы}
\section{Аксиомы теории вероятностей. Вероятностное пространство}
\begin{definition}
  Пусть $\Omega = \{\omega\}$. Набор подмножества $\Omega$ $\mathcal{A}$ называется алгеброй, если
  \begin{enumerate}
    \item $\Omega \in \mathcal{A}$
    \item $\left.
\begin{aligned}
A \subset \Omega, B \subset \Omega \\
A \in \mathcal{A}, B \in \mathcal{A}
\end{aligned}
\right\} \Rightarrow A + B \in \mathcal{A}, A \cdot B \in \mathcal{A}$
  \item $\left.
\begin{aligned}
A \subset \Omega \\
A \in \mathcal{A}
\end{aligned}
\right\} \Rightarrow \overline{A} \in \mathcal{A}$
  \end{enumerate}
\end{definition}
\begin{definition}
  Набор подмножеств исходного множества $\Omega = \{\omega\}$ --- $\mathcal{F}$ называется $\sigma$-алгеброй, если он является алгеброй и:
  \[
    A_1, A_2, \dots, A_i, \dots A_k \subset \Omega \forall k,
  \]
  \[
    A_i \in \mathcal{F} \Rightarrow \sum\limits_{i=1}^\infty A \in \mathcal{F}, \prod\limits_{i=1}^\infty A_i \in \mathcal{F}.
  \]
\end{definition}
\subsection{Аксиомы Колмогорова}
$\big<\Omega, \mathcal{F}, \Prob \big>$ --- измеримое пространство.
\begin{enumerate}
  \item \textbf{Аксиома алгебры событий.} Заданы множества элементарных событий $\Omega = \{\omega\}$. На $\Omega$ выделена $\sigma$-алгебра $\mathcal{F}$, её элементы --- случайные события.
  \item \textbf{Аксиома существования алгебры событий.} Любому случайному событию $A \in \mathcal{F}$ сопоставлено неотрицательное число, называемое вероятностью этого события, $\forall A \in \mathcal{F}: \Prob(A) \geq 0$.
  \item \textbf{Аксиома нормированности.} $\Prob(\Omega) = 1$.
  \item \textbf{Аксиома аддитивности вероятности.} Если $A, B \in \mathcal{F}, A \times B = \emptyset$ $\Rightarrow$ $\Prob(A + B) = \Prob(A) + \Prob(B)$. \\
  \begin{conclusion}
    $A_1, \dots, A_n \in \mathcal{F}$: $A_i \times A_j \not = \emptyset$, $i \not= j$, $i, j = \To n$ $\Rightarrow$ $\Prob(\sum\limits_{i=1}^n A_i) = \sum\limits_{i=1}^n \Prob(A_i)$.
  \end{conclusion}
  \item $A_1, \dots, A_n, \dots, \underset{i \not= j, i,j = 1, 2, \ldots}{A_i \in \mathcal{F} : A_{ij}} = \O$ (попарно несовместны) $\Rightarrow$ $\Prob(\sum\limits_{i=1}^\infty A_i) = \sum\limits_{i=1}^\infty \Prob(A_i)$. \\ $\Prob$ -- нормированная счётно-аддитивная мера
\end{enumerate}
Рассмотрим монотонную случайную последовательность событий $A_1, \dots, A_n, \dots$
\setcounter{equation}{0}
\begin{equation}\label{1-1}
  A_1 \subset A_2 \subset \dots \subset A_n \subset \dots, \overset{\forall n \in \mathbb{N}}{A_n \subset A_{n+1}}
\end{equation}
\begin{equation}\label{1-2}
  A_1 \supset A_2 \supset \dots \supset A_n \supset \dots, \overset{\forall n \in \mathbb{N}}{A_n \supset A_{n+1}}
\end{equation}
Тогда
\[ A = \sum\limits_{i=1}^\infty A_i = \lim\limits_{n \to \infty} A_n \text{--- предел (\ref{1-1})}.
\]
\[ A = \prod\limits_{i=1}^\infty A_i = \lim\limits_{n \to \infty} A_n \text{--- предел (\ref{1-2})}.
\]
\begin{definition}
  Функция событий $Q(A)$ называется непревывной, если для любой монотонной последовательности случайных событий выполняется равенство
  \[
    \lim\limits_{n \to \infty} Q(A_n) = Q(\lim\limits_{n \to \infty} A_n)
  \]
\end{definition}
\begin{enumerate}
  \setcounter{enumi}{4}
  \item[$5'.$] \textbf{Аксиома непрерывности.} Пусть $A_n$ --- монотонно убывающая последовательность случайных событий. \\ $[A_n \downarrow \O] \Leftrightarrow A_{n+1} \subset A_n, n = 1, 2, \dots$ ($\prod\limits_{i=1}^n A_n = \O$ --- предел невозможен). Тогда
  \[
    \lim\limits_{n \to \infty} \Prob(A_n) = 0.
  \]
\end{enumerate}

\begin{theorem}
  Вероятность является непрерывной функцией событий.
\end{theorem}
\begin{proof}
\begin{description}[leftmargin = 0cm]
  \item Из аксиомы (5) $\Rightarrow$ (5') \\
  Пусть $B_n \downarrow \O$, $A_n = B_n \cdot \overline{B_{n+1}}$, где $n = 1,2, \dots$ \\
  \[
    B_1 = \sum\limits_{k=1}^n A_k, B_n = \sum\limits_{k=n}^\infty A_k,
  \]
  \[
    \Prob(B_1) = \Prob(\sum\limits_{k=1}^\infty A_k) = \sum\limits_{k=1}^\infty \Prob(A_k)
  \]

  \[ \Prob(B_n) = \Prob(\sum\limits_{k=n}^\infty A_k) = \underbrace{\sum\limits_{k=n}^\infty \Prob(A_k)}_{n \to \infty} \to 0 \]

  \[ \lim\limits_{n \to \infty} \Prob (B_n) = 0\]
  \item Из аксиомы (5') $\Rightarrow$ (5) \\
  Пусть имеется множество попарно несовместных событий ${\{A_n\}}_{n = 1, 2, \dots}$.
  \[ B_n = \sum\limits_{k=n+1}^\infty A_k, A = \sum\limits_{k=1}^\infty A_k \]
  \[ A = A_1 + \ldots + A_n + B_n \]
  \[ \Prob(A) = \sum\limits_{k=1}^n \Prob(A_k) + \underbrace{\Prob(B_n)}_{\to 0} \]
  \[ B_{n+1} \subset B_n \]
  \[ B_n \Rightarrow A_i, i > n \Rightarrow A_{i+1} \text{ не наступило} \Rightarrow \text{не наступило } B_i, \dots \]
  \[ \prod\limits_{i=1}^\infty B_i = 0 \]
\end{description}
\end{proof}
\section{Свойства вероятности}
\begin{enumerate}
  \item Вероятность невозможного события $\Prob(\O) = 0$
  \begin{proof}
    $ \O+ \Omega = \Omega $
    \[
      1 = \Prob(\Omega) = \Prob(\O + \Omega) \overset{(\text{акс. } 4)}{=} \Prob(\O) + \underbrace{\Prob(\Omega)}_{= 1} = 1
    \]
  \end{proof}
  \item $\Prob(\overline{A}) = 1 - \Prob(A)$
  \begin{proof}
    \[
      1 = \Prob(\Omega) = \Prob(A + \overline{A}) \overset{(\text{акс. } 4)}{=} \Prob(A) + \Prob(\overline{A}) = 1
    \]
  \end{proof}
  \item $A \subset B$ $\Rightarrow$ $\Prob(A) \leqslant \Prob(B)$
  \begin{proof} $B = A \times B + \overline{A} \times B \overset{A \subset B}{=} A + \overline{A} \times B$
    \[
      \Prob(B) = \Prob(A + \overline{A} \times B) \overset{(\text{акс. } 4)}{=} \Prob(A) + \Prob(\overline{A} \times B)
    \]
  \end{proof}
  \item $\Prob(A + B) = \Prob(A) + \Prob(B) - \Prob(A \times B)$
  \begin{proof}
    \[
      A + B = A + \overline{A} \times B, B = A \times B + \overline{A} \times B \Rightarrow
    \]
    \[
      \Prob(A + B) \overset{(\text{акс. } 4)}{=} \Prob(A) + \Prob(\overline{A} \times B)
    \]
    \[
      \Prob(B) = \Prob(A \times B) + \Prob(\overline{A} \times B)
    \]
  \end{proof}
  \item $\Prob(A+B) \leqslant \Prob(A) + \Prob(B)$
  \begin{proof}
    \[
      \Prob(\sum\limits_{k=1}^n A_k) \leqslant \sum\limits_{k=1}^n \Prob(A_k)
    \]
    \[
      \Prob(\sum\limits_{k=1}^\infty A_k) \leqslant \sum\limits_{k=1}^\infty \Prob(A_k)
    \]
  \end{proof}
  \item $A_1, \dots, A_n$:
  \[
    \begin{split}
      \Prob(\sum\limits_{k=1}^\infty A_k) = \sum\limits_{k=1}^\infty \Prob(A_k) - \sum\limits_{k=1}^{n-1} \sum\limits_{i=k+1}^n \Prob(A_k \times A_j) + \ldots \\
      \ldots - (-1)^{n-1} \cdot \Prob(\prod\limits_{k=1}^n A_k)
    \end{split}
  \]
  Пусть $B = \sum\limits_{k=1}^{n+1} A_k$. Тогда
  \[
    \Prob(\sum\limits_{k=1}^{n+1} A_k) = \Prob(A_{k+1} + \sum\limits_{k=1}^n A_k)
  \]
  \item \textbf{Теорема.} Пусть имеется $k$ попарно несовместных и составляющих полную группу благоприятных к событию $A$ исходов из всех исходов $n$, $\{ E_1, \dots, E_n\}_{\underset{i, j = 0, 1, \ldots}{E_i \times E_j = \O, i \not = j}}$, $\sum\limits_{n=1}^n E_i = \Omega$. Тогда
  \[
    \Prob(A) = \frac{k}{n}
  \]
\begin{proof}
  $A = \sum\limits_{s=1}^k E_{i_s}$
  \[
    1 = \Prob(\sum\limits_{i=1}^n E_i) \overset{(\text{акс. } 4)}{=} \sum\limits_{i=1}^n \Prob(E_i) = n \cdot \Prob(E_i), \Prob(E_i) = \frac{1}{n}
  \]
  \[
    \Prob(A) = \Prob(\sum\limits_{s=1}^k E_{i_s}) \overset{(\text{акс. } 4)}{=} \sum\limits_{s=1}^k \Prob(E_{i_s}) = \frac{k}{n}
  \]
  \[
    \Omega = \{E_1, \dots, E_n\}, \mathcal{F} = \{A = E_{i_1} + \ldots + E_{i_k}\}, k \leqslant n
  \]
  \[
    \Prob(A) = \frac{k}{n}
  \]
\end{proof}
  \setcounter{enumi}{7}
  \item $0 \leqslant \Prob(A) \leqslant 1, A \in \mathcal{F}$
  \[
    0 \leqslant \Prob(\overline{A}) = 1 - \Prob(A) \Rightarrow \Prob(A) \leqslant 1
  \]
  Для условных вероятностей:
  \[
    \Prob(A|B) \overset{def}{=} \frac{\Prob(A \times B)}{\Prob(B)}
  \]
  $\Prob(B) > 0$
  \[
    \Prob(\Omega | B) = \frac{\Prob(\Omega \times B)}{\Prob(B)} = \frac{\Prob(B)}{\Prob(B)} = 1.
  \]
  Аксиома нормированности выполнена. \\
  Пусть $A_1, \ldots, A_n, \ldots$ --- не более чем счётный набор, $A_i \times A_j = \O, i \not = j; i,j = \To n$ \\
  Пусть также имеется событие $B$, $\Prob(B) > 0$.
  \[
    \Prob(\sum\limits_{n} A_n | B) = \frac{\Prob((\sum\limits_{n} A_n) \times B)}{\Prob(B)} = \frac{\sum\limits_{n} \Prob(A_n \times B_n)}{\Prob(B)} =
  \]
  \[
    \overset{(\text{акс. } 4, 5)}{=} \sum\limits_{n} \Prob(A_n | B)
  \]
\end{enumerate}
