\part{Характеристические функции. Предельные теоремы}
\setcounter{equation}{0}
\section{Производящие функции. Факториальные моменты}
Пусть задана случайная величина $\xi$, $m = 0, 1, 2, \ldots$, закон распределения, которому подчиняется $\xi$: 
\[
	\Prob \{ \xi = m \} = p_m, \sum\limits_{m = 0}^{\infty} p_m = 1
\]
Такой закон удобно исследовать с помощью производящих функций. Пусть $u \in \mathbb{R}^1$. Определим производящую функцию дискретной случайной величины
\begin{equation}
	\Psi_{\xi} (u) \underset{\textrm{def} }{=} \MExpect_u^{\xi} = \sum\limits_{m} p_m \cdot u^m, \ |u| \leqslant 1
\end{equation}
Рассмотрим данный ряд. Он абсолютно сходится для $|u| \leqslant 1$
\begin{equation}
	p_m = \frac{1}{m!} \left. \frac{d^m \Psi_{\xi} (u)}{du^{m}} \right|_{u = 0}
\end{equation}
\[
	\xi : \{ p_k \}, \Psi_{\xi} (u), \Psi_{\xi}(1) = 1
\]
Существует взаимно однозначное соответствие между производящими функциями и соответствующими законами распределения
\begin{theorem}
	Пусть задан набор целочисленных неотрицательных независимых случайных величин $\xi_1, \ldots, \xi_n$. Обозначим $\xi_k \sim \Psi_{\xi_k} (u)$, то есть каждому элементу соответствует производящая функция. Тогда
\[
	\Psi_{\xi_1 + \ldots + \xi_n} (u) = \prod\limits_{k = 1}^{n} \Psi_{\xi_k} (u)
\]
\end{theorem}
\begin{proof}
	$u^{\xi_1}, \ldots, u^{\xi_n}$ --- независимы, поскольку $\xi_1, \ldots, \xi_n$ независимы. $g(x) = a^x$
\[
	\MExpect_{u^{\xi_1 + \ldots + \xi_n}} = \MExpect_{u^{\xi_1} \cdot \ldots \cdot u^{\xi_n}} = \prod\limits_{k = 1}^{n} \MExpect_{u^{\xi_k}}
\]
\end{proof}
\begin{example}
	Рассмотрим закон распределения Бернулли $B[n, p]$. $\xi_i$ --- число появления успеха в $i$-ом испытании.
\[
	\mu_n = \sum\limits_{i = 1}^{n} \xi_i, \ \Psi_{\mu_n} (u) = \prod\limits_{k = 1}^{n} \Psi_{\xi_k} (u)
\]
\[
	\Psi_{\xi_k} (u) = \MExpect_{u^{\xi_k}} = u^0 q + u^1 p, \ \Psi_{\mu_n} (u) = (pu + q)^n
\]
\[
	\Prob \{ \xi + \eta = n \} = \sum\limits_{k = 0}^{n} \Prob \{ \xi = k \} \cdot \Prob \{ \eta = n - k \}
\]
Используя теорему 1 можно найти композицию (свёртку) распределения, не прибегая к формуле свёртки.
\end{example}
\begin{theorem}
	Пусть задан набор целочисленных неотрицательных независимых одинаково распределённых случайных величин $\xi_1, \ldots, \xi_n$
\[
	\forall k: \xi_k \sim \Psi_{\xi} (u)
\]
\[
	\left.
	\begin{aligned}
	\eta_{\nu} = \xi_1 + \ldots + \xi_{\nu}, \nu \geqslant 1 \\
	\eta_{\nu} = 0, \nu < 1
	\end{aligned}
	\right	\} \Rightarrow \Psi_{\eta_{\nu}} (u) = \Psi_{\nu} (\Psi_{\xi} (u))
\]
\end{theorem}
\begin{proof}
\[
	\MExpect [u^{\xi_1 + \ldots + \xi_{\nu}} \ | \ \nu = n] = \MExpect_{u^{\xi_1 + \ldots + \xi_n}} = \left[ \Psi_{\xi} (u) \right]^n
\]
$n \in \mathbb{N}$
\[
	\Psi_{\eta_{\nu}} (u) = \underset{\textrm{def}}{=} \MExpect_{u^{\eta_{\nu}}} = \MExpect \left[ \MExpect \left[ u^{\eta_{\nu}} \ | \ \nu \right] \right] = \MExpect \left[ \left[ \Psi_{\xi} (u) \right]^{\nu} \right] = \Psi_{\nu} \left( \Psi_{\xi} (u)\right)
\]
\end{proof}
\begin{definition}
	$k$-ым факториальным моментом целочисленной неотрицательной случайной величины $\xi$ называется математическое ожидание $\MExpect_{\xi}^{[k]}$, такое, что
    \[
	    \xi^{[k]} = \xi (\xi - 1) \ldots (\xi - k + 1)
    \] 
	\[
		m^{[k]} = m (m - 1) \ldots (m - k + 1)
	\]
	\[
		m^{[k]} = 0, \ m < k
	\]
	При $k = 0$: $\xi^{[0]} = 1$
	\[
		\MExpect_{\xi^{[1]}} = \MExpect_{\xi}, \ \MExpect_{\xi^{[2]}} = \MExpect_{\xi^{[2]}} - \MExpect_{\xi}
	\]
	\[
		\Variance_{\xi} = \MExpect_{\xi^{[2]}} + \MExpect_{\xi^{[1]}} - (\MExpect_{\xi^{[1]}})^2
	\]
\end{definition}
\begin{theorem}
	Если существует факториальный момент $k$-ого порядка $\MExpect_{\xi^{[k]}}$, то существует левосторонняя $k$-ая производная производящей функции
	\[
		\exists \MExpect_{\xi^{[k]}} \Rightarrow \exists \Psi_{\xi}^{(k)} (1 - 0), \ \Psi_{\xi}^{(k)} (1 - 0) = \MExpect_{\xi^{[k]}}
	\]
$|u| < 1$
\[
	\Psi_{\xi}^{(k)} (u) = \sum\limits_{m = k}^{\infty} m^{[k]} \cdot u^{m - k} \Prob \{ \xi = m \}
\]
По второй теореме Абеля:
	\[
		\MExpect_{\xi^{[k]}} = \sum\limits_{m = k}^{\infty} m^{[k]} \Prob \{ \xi = m \}
	\]
\end{theorem}
\begin{theorem}
	\textit{(Абеля)} Пусть $r > 0$, тогда
	\[
		\sum\limits_{k = 0}^{\infty} a_k r^k = S
	\]
	\[
		\sum\limits_{k = 0}^{\infty} a_k x^k, x \in [0, r]
	\]
	\[
		\lim\limits_{x \to r - 0} \sum\limits_{k = 0}^{\infty} a_k x^k = S
	\]
\end{theorem}
Оказывается, что соответствие между рассмотренными законами распределения вероятностей и производящими функциями не только взаимно однозначно, но ещё и взаимно непрерывно.
\begin{theorem} \textit{(Непрерывность производящих функций)} 
	Пусть при фиксированных $n$ $(n = 1, 2, \ldots)$:
	\[
		{\{ p_k (n) \}}_{k = 0, 1, 2, \ldots} : p_k (n) \geqslant 0, \forall k
	\]
	\[
		\sum\limits_{k = 0}^{\infty} p_k (n) = 1
	\]
	\[
		\Psi_m (u) = \sum\limits_{k = 0}^{\infty} p_k (n) u^k
	\]
	\[
		\lim\limits_{n \to \infty} p_k (n) = p_k, \ \sum\limits_{k = 0}^{\infty} p_k = 1 \Leftrightarrow \forall 0 < u \leqslant 1: \lim\limits_{n \to \infty} \Psi_n (u) = \Psi (u),
	\]
	где $\Psi (u) = \sum\limits_{k = 0}^{\infty} p_k u^k$
\end{theorem}
\begin{example}
	Рассмотрим биномиальное распределение. $\mu_n, p_n$
	\[
		\lim\limits_{n \to \infty} n p_n = a, \ \lim\limits_{n \to \infty} \Prob \{ \mu_n = m \}
	\]
	\[
		\Psi_{\mu_n} (u) = \sum\limits_{m = 0}^{n} \Prob \{ \mu_n = m \} \cdot u^m = \left( \frac{a_n}{n} \cdot u + 1 - \frac{a_n}{n} \right)^n =
	\]
	\[
		= \left( 1 + \frac{a_n}{n} (u - 1) \right)^n \underset{n \to \infty}{\rightarrow} e^{a(u - 1)} = \sum\limits_{m = 0}^{\infty} \frac{a}{m!} e^{-a} u^m,
	\]
	То есть
	\[
		\Prob \{ \mu_n = m \} \underset{n \to \infty}{\rightarrow} \frac{a^m}{m!} e^{-a}
	\]
\end{example}
Рассмотрим случайный вектор $\overline{\xi} = (\xi_1, \ldots, \xi_n)$, где $\xi_i$ --- целочисленная непрерывная случайная величина. Также введём вектор значений $\overline{m} = (m_1, \ldots, m_n)$, не более чем счётный набор. То есть
\[
	\Prob_{\overline{m}} = \Prob \{ \overline{\xi} = \overline{m} \}
\]
Введём производящую функцию:
\[
	\Psi_{\overline{\xi}} (u_1, \ldots, u_m) \overset{\textrm{def}}{=} \MExpect [u_1^{\xi_1} \cdot \ldots \cdot u_m^{\xi_m}] = \sum\limits_{\overline{m}} \Prob_{\overline{m}} \cdot u^{m_1} \cdot \ldots \cdot u^{m_n}
\]
Можем определить смешанный факториальный момент порядка $k_1 + \ldots + k_n, \ k_i \geqslant 0, \ i = 1, 2, \ldots n$
\[
	\MExpect_{\xi_1^{[k_1]} \cdot \ldots \cdot \xi_n^{[k_n]}}, \xi_i^{[k]} = \xi_i (\xi_i - 1) \ldots (\xi_i - k + 1), \ \xi^{[0]} = 1
\]
\[
	\MExpect_{\xi_1^{[k_1]} \cdot \ldots \cdot \xi_n^{[k_n]}} = 
\left. \frac{ \partial^{k_1 + \ldots + k_n} \Psi_{\overline{\xi}} (u_1, \ldots, u_m) }{ \partial u_1^{k_1} \cdot \ldots \cdot \partial u_n^{k_n} } \right|_{u_1 = \ldots = u_n = 1}
\]

\section{Характеристические функции случайных величин}
Пусть $\xi$, $\eta$ --- случайные величины, $i: \ i^2 = -1$. Составим случайную величину $\theta = \xi + i \eta$. \\
Пусть также существуют математические ожидания введённых величин: $\MExpect_{\xi}$, $\MExpect_{\eta}$. Тогда можем составить математическое ожидание комплексно-значной случайной величины:
\[
	\MExpect_{\theta} = \MExpect_{\xi} + i \MExpect_{\eta} 
\]
Все свойства математического ожидания вещественно-значной случайной величины переносятся также на комплексно-значную случайную величину. \\
$\theta_1$ и $\theta_2$ независимы, если независимы два вектора $(\xi_1, \eta_1)$ и $(\xi_2, \eta_2)$.
\begin{definition}
	Характеристической функцией вещественно-значной случайной величины называется функция вещественного агумента, которая представляет собой математическое ожидание $e^{it\xi}$:
\[
	\phi_{\xi} (t) = \MExpect_{e^{it\xi}}, \ t \in \mathbb{R}^1, \text{ $\xi$ --- вещественная случайная величина}
\]
\[
	e^{i \alpha} = cos \alpha + i sin \alpha
\]
\[
	\phi_{\xi} (t) = \MExpect [cos \xi t] + i \MExpect_ [sin \xi t]
\]
\[
	|e^{it\xi}| = 1, \ \theta = a + i b, \ |\theta| = \sqrt{a^2 + b^2}
\]
\[
	\phi_{\xi} (t) = \int\limits_{-\infty}^{+\infty} e^{itx} d \mathcal{P}_{\xi} (x)
\]
\end{definition}
Характеристическая функция полностью определяется распределением своей случайной величины.
\[
	\exists \MExpect_{\xi}^{[k]} \Rightarrow \exists \psi_{\xi}^{[k]} (1 - 0), \ \psi_{\xi}^{(k)} (1 - 0) = \MExpect_{\xi}^{[k]}
\]
В случае, если $\xi$ --- абсолютно-непрерывная случайная величина
\[
	\phi_{\xi} (t) = \int\limits_{-\infty}^{+\infty} e^{itx} f_{\xi} (x) dx
\]
Если $\xi$ --- дискретная случайная величина:
\[
	\phi_{\xi} (t) = \sum\limits_{k} e^{itx} \cdot \Prob \{ \xi = x_k \},
\]
где $\{ x_k \}_{k = 1, \ldots}$ --- не более чем счётный набор
\subsection{Простейшие свойства характеристической функции}
\begin{enumerate}[wide, labelwidth=!, labelindent=0pt]
	\item $|\phi(t)| \leqslant 1, \ \forall t \in \mathbb{R}^1, \ \phi(0) = 1$ \\
\[ | \MExpect_{e^{it \xi}} | \leqslant \MExpect_{|e^{it \xi}|} = 1 \]
	\item $\xi$ --- случайная величина, $a, b$ --- константы, $\eta = a \xi + b$
\[
	\phi_{\eta} (t) = \MExpect_{e^{it(a \xi + b)}} = e^{it b} \cdot \phi_{\xi} (at)
\]
$c$ --- константа
\[
	\phi_c (t) = e^{itc}
\]
	\item $\xi_1, \ldots, \xi_n$ --- вектор независимых случайных величин. Тогда
\[
	\phi_{\xi_1 + \ldots + xi_n} (t) = \prod\limits_{k = 1}^{n} \phi_{\xi_k} (t) \Rightarrow e^{it \xi_1}, \ldots, e^{it \xi_n}
\]
\item $\phi_{\xi} (-t) = \phi_{- \xi} (t) = \overline{\phi_{\xi} (t)}$
\item $\xi \geqslant 0$, целочисленный случайный вектор. Тогда производящая функция $\psi_{\xi} (u) = \MExpect_{u^{\xi}}$. Тогда
\[
	\phi_{\xi} (t) = \MExpect_{e^{it\xi}} = \psi_{\xi} (e^{it})
\]
\item Характеристическая функция равномерно непрерывна по аргументу $(t)$ на всей числовой оси
\[
	| \phi(t + h) - \phi(t) | = \Delta (h) \underset{h \to 0}{\rightarrow} 0, \ \forall t \in \mathbb{R}^1
\]
Покажем это.
\[
	\phi(t) = \int\limits_{-\infty}^{+\infty} e^{itx} d \mathcal{P}_{\xi} (x)
\]
\[
	| \phi (t + h) - \phi (t) | \leqslant \int\limits_{-\infty}^{+\infty} | e^{i (t + h) x} - e^{itx} | d \mathcal{P}_{\xi} (x) =
\]
Известно, что
\[
	| \theta_1 \cdot \theta_2 | = |\theta_1| \cdot | \theta_2 |, \ | e^{itx} | = 1
\]
\[
	= \int\limits_{-\infty}^{+\infty} \underbrace{| e^{ihx} - 1 |}_{g_h (x)} d \mathcal{P}_{\xi} (x) = \Delta (h) \underset{h \to 0}{\rightarrow} 0
\]
$g_h(x) \underset{h \to 0}{\rightarrow} 0$, $g_h(x) \leqslant 2, \forall x \in \mathbb{R}^1$
\subsection{Примеры}
\begin{itemize}
	\item $B [n, p]$, $q = 1 - p$
\[
	\phi_{\xi} (t) = (pe^{it} + q)^n
\]
	\item $\xi \sim P [a]$, $a > 0$, $e^{a(u - 1)}$
\[
	\phi_{\xi} (t) = exp \{ a(e^{it} - 1) \}
\]
    \item $\xi = c$ с вероятностью 1.
\[
	\phi_c (t) = e^{itc}
\] 
	\item $\xi \sim N(0, 1)$. Доказать самостоятельно:
\[
	\phi_{\xi} (t) = e^{- \frac{t^2}{2}}
\]
\item $\eta \sim N(a, \sigma)$, $a \in \mathbb{R}^1$, $\theta > 0$
\[
	\phi_{\eta} (t) = е^{{ita} - \frac{\sigma^2 t^2}{2}}
\]
\item $\xi \sim R [a, b]$
\[
	\phi_{\xi} (t) = \frac{e^{i t b} - e^{i t a}}{it(b - a)}
\]
\end{itemize}
\begin{theorem}
	\textit{(Формула обращения)} Пусть $\xi$ --- случайная величина, задана функция распределения $F(x)$, характеристическая функция $\phi(t)$. Пусть также $x_1 < x_2$ --- точки непрерывности $F(x)$. Тогда имеет место соотношение
\[
	F(x_2) - F(x_1) = \frac{1}{2 \pi} \cdot \lim\limits_{A \to + \infty} \int\limits_{-A}^{A} \frac{e^{-it x_1} - e^{-it x_2}}{it} \phi(t) dt
\]
\end{theorem}
\begin{proof}
\[
	\int\limits_{-\infty}^{+\infty} | \phi_{\xi} (t) | dt < \infty
\]
\[
	f_{\xi} (x) = \frac{1}{2 \pi} \int\limits_{-\infty}^{+\infty} e^{it x} \cdot \phi(t) dt, \ a < b
\]
Проинтегрируем по $x$:
\[
	\int\limits_{a}^{b} f_{\xi} (x) dx = F(b) - F(a) = \int\limits_{a}^{b} \left[ \int\limits_{-\infty}^{+\infty} e^{-itx} \cdot \phi(t) dt \right] dx \overset{\text{т. Фубини}}{=}
\]
\[
	= \frac{1}{2 \pi} \int\limits_{-\infty}^{+\infty} \phi(t) \left[ \int\limits_{a}^{b} e^{itx} dx \right] dt = \frac{1}{2 \pi} \int\limits_{-\infty}^{+\infty} \frac{e^{-ita} - e^{-itb}}{it} \phi(t) dt
\]
\end{proof}
\begin{theorem}
	Функция распределения однозначно определяется характеристической функцией своей случайной величины. 
\end{theorem}
Из предыдущей теоремы следует, что для любой точки непрерывности функции распределения соответствует формула обращения. Пусть $x$ --- точка непрерывности $F(x)$. Тогда
\[
	F(x) = \frac{1}{2 \pi} \lim\limits_{y \to -\infty} \lim\limits_{A \to +\infty} \int\limits_{-A}^{A} \frac{e^{-ity} - e^{-itx}}{it} \phi(t) dt
\]
\[
	F_{\xi} (x) = \lim\limits_{x_1 \downarrow x} F(x_1)
\]
\begin{example} 
	\begin{itemize}
		\item Пусть $\xi \sim N(a_1, \sigma_1)$, $\xi_2 \sim N(a_2, \sigma_2)$, где $\xi_1, \xi_2$ --- независимые случайные величины. Каков закон распределения $\xi_1 + \xi_2$?
\[
	\phi_{\xi_k} (t) = e^{it a_k - \frac{\sigma^2_k t^2}{2}}, \ k = 1, 2, \ldots
\]
\[
	\phi_{\xi_1 + \xi_2} (t) = e^{it(a_1 + a_2) - \frac{(\sigma_1^2 + \sigma_2^2)t^2}{3}}
\]
\[
	\xi_1 + \xi_2 \sim N(a_1 + a_2, \sqrt{\sigma_1^2 + \sigma_2^2})
\]
\item $\xi_n^2 = \xi_1^2 + \ldots + \xi_n^2$, $\xi_k \sim N (0, 1)$, $k = 1, \ldots, n$, $\xi_1, \ldots,\xi_n$ --- независимые случайные величины
	\end{itemize}
\end{example}
\begin{theorem}
	Пусть математическое ожидание некоторой случайной величины конечно $\MExpect_{|\xi|^n}$. Тогда характеристическая функция $\xi$ дифференцируема $n$ раз и справедливо:
\[
	\phi_{\xi}^{(k)} (0) = i^k \MExpect_{\xi^k}, \ k \leqslant n
\]
\end{theorem}
Покажем это.
\[
	\phi_{\xi}^{(k)} (t) = i^k \int\limits_{-\infty}^{+\infty} e^{itx} \cdot x^k d \mathcal{P}_{\xi}
\]
\[
	\left| \int\limits_{-\infty}^{+\infty} x^k \cdot e^{itx} d \mathcal{P}_{\xi} \right| \leqslant \int\limits_{-\infty}^{+\infty} |x|^k d \mathcal{P}_{\xi} < \infty
\]
Рассмотрим логарифм характеристической функции
\[
	\kappa (t) = ln(\phi(t)), \ \kappa'(t) = \frac{\phi'(t)}{\phi(t)}
\]
\[
	\kappa''(t) = \frac{\phi''(t) \cdot \phi(t) - [\phi'(t)]^2}{[\phi(t)]^2}
\]
\[
	\kappa'(0) = \phi'(0) = i \cdot \MExpect_{\xi}
\]
\[
	\kappa''(0) = \phi''(0) - [\phi'(0)] = - \Variance_{\xi}
\]
\begin{definition}
	Производная $k$-ого порядка логарифма характеристической функции случайной величины $\xi$ в нуле, умноженная на $i^k$, называется семиинвариантом $k$-ого порядка случайной величины $\xi$
\end{definition}
Семиинвариант суммы независимых случайных величин равен сумме семиинвариантов слагаемых того же порядка. \\
Семиинвариант любого порядка есть рациональная функция моментов порядков не превосходящих $k$. \\
При сложении независимых случайных величин их функции распределения свёртываются, характеристические перемножаются, семиинварианты складываются.

\item test
\[
	\phi(t) = \sum\limits_{k = 0}^{n} \frac{(it)^k}{k!} \MExpect_{\xi}^k + o(|t|^n)
\]
\[
	\phi(t) = \phi(0) + t \cdot \phi'(0) + \frac{t^2}{2} \phi''(0) + o(t^2) = 
\]
\[
	= 1 + it \cdot \MExpect_{\xi} - \frac{t^2}{2} \MExpect_{\xi^2} + o(t^2)
\]

\begin{theorem}
	\[
		\forall x : F_n (x) \underset{n \to \infty}{\rightarrow} F(x) \Leftrightarrow \phi_n (t) \underset{n \to \infty}{\rightarrow} \phi(t)
	\]
где $x$ --- точка непрерывности $F(x)$. \\
Чтобы последовательность функций распределения $F_n$ сходилась в каждой точке к предыдущей функции распределения, необходимо и достаточно, чтобы последовательность соответствующих функций в каждой точке $t$ сходилась к характеристической функции, соответствующей этой функции распределения. \\
Если $\phi$ --- вещественная, то она чётная.
\end{theorem}
\end{enumerate}



























