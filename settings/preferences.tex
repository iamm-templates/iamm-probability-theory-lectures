\title{Теория вероятностей. Конспект}
\author{В. А. Тюльпин}

\addto\captionsrussian{
  \renewcommand{\partname}{Глава}
  \renewcommand{\contentsname}{Содержание}
  \renewcommand{\refname}{Перечень использованных источников}
  \renewcommand{\bibname}{Перечень использованных источников}
}

\newcommand{\university}{САНКТ-ПЕТЕРБУРГСКИЙ ПОЛИТЕХНИЧЕСКИЙ \\ УНИВЕРСИТЕТ ПЕТРА ВЕЛИКОГО}
\newcommand{\faculty}{Институт прикладной математики и механики}
\newcommand{\department}{ }
\newcommand{\city}{Санкт-Петербург}
\newcommand{\type}{Теория вероятностей. Kонспект.}
\newcommand{\num}{ № 1}
\newcommand{\subject}{Теория вероятностей. Конспект}
\newcommand{\tutorname}{к.б.н. Н. О. Кадырова}
\newcommand{\studentname}{В. А. Тюльпин,  Ю. А. Камалетдинова}
\newcommand{\course}{3 курс}
\newcommand{\group}{33633/1}

\renewcommand{\thesection}{\arabic{section}}
\titleformat{\section}{\normalfont\Large\bfseries}{\S\thesection}{1em}{}
\setlength{\parindent}{0pt}
\bibliographystyle{ugost2008}
\providecommand*{\bblmay}[1]{#1}
\graphicspath{ {} }

%% Перенос знаков в формулах (по Львовскому)
\newcommand*{\hm}[1]{#1\nobreak\discretionary{}
{\hbox{$\mathsurround=0pt #1$}}{}}
%\usepackage{setspace}
%\singlespacing

\setcounter{tocdepth}{1} 
\makeatletter
\@addtoreset{section}{part}
\makeatother
\pagestyle{plain}
\flushbottom

\theoremstyle{definition}
\newtheorem*{theorem}{Теорема}
\theoremstyle{definition}
\newtheorem*{lemma}{Лемма}
\theoremstyle{definition}
\newtheorem*{definition}{Определение}
\theoremstyle{definition}
\newtheorem*{conclusion}{Следствие}
\theoremstyle{definition}
\newtheorem*{example}{Пример}
\theoremstyle{definition}
\newtheorem*{addition}{Замечание}
\theoremstyle{definition}
\newtheorem*{interjection}{Отступление}

\newcommand{\MExpect}{\mathsf{M}}
\newcommand\To[1]{1,2,\dots,#1}
\newcommand\ZTo[1]{0,1,\dots,#1}
