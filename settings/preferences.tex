% настройки polyglossia
\setdefaultlanguage{russian}
\setotherlanguage{english}

% локализация
\addto\captionsrussian{
  \renewcommand{\figurename}{Рисунок}%
  \renewcommand{\partname}{Глава}
  \renewcommand{\contentsname}{\centerline{Содержание}}
  \renewcommand{\listingscaption}{Листинг}
}

% основной шрифт документа
\setmainfont{CMU Serif}

% перечень использованных источников
% \addbibresource{refs.bib}

% настройка полей
\geometry{top=2cm}
\geometry{bottom=2cm}
\geometry{left=2cm}
\geometry{right=2cm}
\geometry{bindingoffset=0cm}

% настройка ссылок и метаданных документа
\hypersetup{unicode=true,colorlinks=true,linkcolor=red,citecolor=green,filecolor=magenta,urlcolor=cyan,        		       
    pdftitle={\docname},   	    
    pdfauthor={\studentname},      
    pdfsubject={\subject},      		        
    pdfcreator={\studentname}, 	       
    pdfproducer={Overleaf}, 		     
    pdfkeywords={\subject}
}

% настройка подсветки кода и окружения для листингов
\usemintedstyle{colorful}
\newenvironment{code}{\captionsetup{type=listing}}{}

% шрифт для листингов с лигатурами
\setmonofont{[FiraCode-Regular.otf]}[
  Contextuals=Alternate  % Activate the calt feature
]

% оформления подписи рисунка
\captionsetup[figure]{labelsep = period}

% подпись таблицы
\DeclareCaptionFormat{hfillstart}{\hfill#1#2#3\par}
\captionsetup[table]{format=hfillstart,labelsep=newline,justification=centering,skip=-10pt,textfont=bf}

% путь к каталогу с рисунками
\graphicspath{{fig/}}

\titlelabel{\thetitle.\quad}

% для удобного конспектирования математики
\mathtoolsset{showonlyrefs=true}

\theoremstyle{definition}
\newtheorem*{theorem}{Теорема}
\theoremstyle{definition}
\newtheorem*{lemma}{Лемма}
\theoremstyle{definition}
\newtheorem*{definition}{Определение}
\theoremstyle{definition}
\newtheorem*{conclusion}{Следствие}
\theoremstyle{definition}
\newtheorem*{example}{Пример}
\theoremstyle{definition}
\newtheorem*{addition}{Замечание}
\theoremstyle{definition}
\newtheorem*{interjection}{Отступление}

% настоящее матожидание
\newcommand{\MExpect}{\mathsf{M}}

% объявили оператор!
\DeclareMathOperator{\sgn}{\mathop{sgn}}

% перенос знаков в формулах (по Львовскому)
\newcommand*{\hm}[1]{#1\nobreak\discretionary{} {\hbox{$\mathsurround=0pt #1$}}{}} 

\renewcommand{\thesection}{\arabic{section}}

\titleformat{\section}{\normalfont\Large\bfseries}{\S\thesection}{1em}{}
\setlength{\parindent}{0pt}

\setcounter{tocdepth}{1} 

\makeatletter
\@addtoreset{section}{part}
\makeatother
\pagestyle{plain}
\flushbottom

% 
\newcommand\To[1]{1,2,\dots,#1}
\newcommand\ZTo[1]{0,1,\dots,#1}

\newcommand*{\textitplusparen}[1]{\textit{#1)}}